\documentclass{rmaa}
%\documentclass[manuscript]{rmaa}
\usepackage{amsmath}

\title{Velocity dispersion of NGC-6366}

\author{V. Lora$^1$ and F. J. S\'anchez-Salcedo$^2$ 
\affil{$^1$Instituto de Radioastronom\'ia y Astrof\'isica UNAM, M\'exico \\
$^2$Instituto de Astronom\'ia UNAM, M\'exico}}

\fulladdresses{
\item V. Lora: Instituto de Radioastronom\'ia y Astrof\'isica UNAM,
Ap. 3-72, 58089 Morelia Michoac\'an, M\'exico
(v.lora@irya.unam.mx)

\item F. J. S\'anchez-Salcedo:
Instituto de Astronom\'ia UNAM, 
Ap. 70-543, 04510 D. F., M\'exico
(jsanchez@astro.unam.mx)}

\shortauthor{Lora \& Sanchez-Salcedo}
\shorttitle{velocity dispersion of NGC-6822}

\SetVolume{333} \SetFirstPage{333} \SetYear{2013}
\ReceivedDate{\today} 
\AcceptedDate{Year Month Day} 

\resumen{En este trabajo proponemos un modelo anali\'itico para un c\'umulo globular 
con un hoyo negro central, de masa intermedia. Este modelo consiste de la soluci\'on 
a una ecuaci\'on de Lane-Emden modificada, donde asumimos que la velocidad del sonido 
isot\'ermica, es proporcional a la velocidad (cuya dependencia es radial) circular orbital.
Comparamos este modelo anal\'itico con simulaciones de $N$-cuerpos (de un c\'umulo con 
un objeto masivo central), mostrando que se reproducen las distribuciones de densidades 
num\'ericas obtenidas. El modelo anal\'itico provee una manera sencilla de estimar la masa
de un hoyo negro central, de observaciones de la estratificaci\'on de densidad radial de 
c\'umulos globulares.}

\abstract{The globular cluster NGC-6822 has a peculiar...}

\keywords{Galaxy: stellar clusters -- kinematics and dynamics -- numerical simulations}

\begin{document}

\maketitle

\section{Introduction}
The Milky Way (MW) galaxy hosts  


The paper is organized as follows.
In \S \ref{LE} we deduce the modified Lane-Emden equation for a GC with a
central BH, and derive the analytic solution for the density and
mass distributions. In \S \ref{nbody} we describe the $N$-body code that has
been used. The numerical
simulations of a GC and a central BH are presented in \S \ref{simulations}.
In \S \ref{comparison} we show a comparison between the numerical
simulations and the analytic model. Finally, we present our conclusions in
\S\ref{conclusions}.

\section{The globular cluster NGC-6366}
\label{ngc-6366}

The GC NGC-6366 is one of the closest GC to the Sun ($R_{_\odot}=3.6$~kpc), 
and it is located at a Galactocentric distance of $\sim5$~kpc \citep{chen:10}. 
In the new GAIA DR2 release \citep{helmi:18}, they report a position on the sky 
for NGC-6366 $(\alpha , \delta)=(261.9393,-5.0752)�$.
This GC is classified as metal rich with $[Fe/H]=-0.55$ \citep{puls:18}, and it 
is very possible associated with the MW bulge. Bulge GCs usually have 
$[Fe/H]>-1.3$ \citep{bica:16}. Another interesting feature of bulge GCs is, that 
they frequently pass through high density regions in the Galaxy, therefore they 
are subject of tidal stripping. The latter seems to be the case for NGC-6366. Its 
location in the inner parts of the Galaxy makes it likely to experience frequent 
gravitational interactions, which could lead to the depletion of low mass stars 
\citep{paust:09}. Furthermore, \cite{gnedin:97} compute a life time for NGC-6366 
in the range of $1.15-6.3$~ Gyr. To compute the life time of the Galactic GCs, they 
they take into account the evaporation of stars from the clusters, and the tidal
shock from the Galactic bluge and disk.




% \begin{figure}[!t]
% \centering
% \includegraphics[scale=0.7]{fig1.eps}
% \caption{Density (top) and circumscribed mass (bottom)
% as a function of spherical radius $R$, obtained
% from the modified Lane-Emden equation with different
% values of the dimensionless parameter $\alpha$ (see
% equation \ref{a}). The solutions for $\alpha=0.3$
% (close to the lower limit of 0.25 for this parameter),
% 0.5, 1.0, 1.5 and 2.5 are shown with the line types
% shown in the bottom right of the figure. The radius
% is given in units of $R_2$ (the radius at which
% the cluster mass is equal to the mass $M_c$ of the
% central object), the density in units of $\rho_2$
% (the density at $R_2$) and the mass in units of
% $M_c$.}
% \label{fig1}
% \end{figure}

The $\alpha=1$ solution has a flat core, and for
decreasing values of $\alpha$ the density stratifications
have increasingly higher central ``cusps''. For values
of $\alpha$ approaching the $\alpha=1/4$ lower limit,
the density stratification approaches a $\rho\propto R^{-3}$
law (see Figure 1 and equation \ref{rhor}).

\section{The N-body code}
\label{nbody}

We carried out $N$-body simulations of a star cluster and a central


\section{The numerical simulations}
\label{simulations}

For the initial mass density profile of the stellar cluster, we used a 
Plummer \citep{plummer11} mass density profile given by
\begin{eqnarray}
\rho_{c}(r) &=& \rho_{0} (1+r^2/r_p^2)^{-5/2} \mbox{ ,}
\label{eq:plummer}
\end{eqnarray}
where $r_p$ is the Plummer radius. We normalized the physical units of 
our simulations such that $r_p=1$ and the total mass of the star cluster 
is $M_{*}=1$.
We first carried out an $N$-body simulation with the code described in 
section
\ref{nbody} using the initial condition given by equation \ref{eq:plummer}.

To the equations of motion in the $N$-body code, we then added the contribution
of the potential of a massive object $M_c$ located at the center of the stellar
cluster. We then carried out two simulations, model M1 with $M_c=0.1\,M_*$ and
model M2 $M_c=0.5\,M_*$ (where $M_*$ is the mass of the cluster). Since the
cluster is modeled with $10^3$ equal mass particles, these values of $M_c$
correspond to 100 and 500 times the mass of the individual particles.
We let the two models (M1 and M2) evolve from $t=0$ to $\sim 15$ crossing
times.


% \begin{figure}[!t]
% \centering
% \includegraphics[scale=0.65]{fig2.eps}
% \caption{Density (top) and circumscribed mass (bottom)
% as a function of radius obtained from models M1
% ($M_c=0.1\,M_*$, short dashes) and M2
% ($M_c=0.5\,M_*$, long dashes). The solid lines correspond
% to the analytic model (equations \ref{rhor} and \ref{mr})
% with $\alpha=0.47$ (see equation \ref{a}).}
% \label{fig2}
% \end{figure}


\section{Conclusions}
\label{conclusions}

\acknowledgments
VL gratefully acknowledges
FJSS gratefully acknowledges

\begin{thebibliography}

\bibitem[\protect\citeauthoryear{Bica et al.}{2016}]{bica:16}
Bica, E., Ortolani, S. \& Barbuy, B., 2016, PASA, 33, e028

\bibitem[\protect\citeauthoryear{Chen \& Chen}{2010}]{chen:10}
Chen, C. W. \& Chen, W. P. 2010, ApJ, 721, 1790

\bibitem[\protect\citeauthoryear{Helmi et al.}{2018}]{helmi:18}
Gaia Collaboration; Helmi, A. et al. 2018, A\&A, XXX

\bibitem[\protect\citeauthoryear{Gnedin \& Ostriker}{1997}]{gnedin:97}
Gnedin, O. Y. \& Ostriker, J. P., 1997, ApJ, 474, 223

\bibitem[\protect\citeauthoryear{Paust et al.}{2009}]{paust:09}
Paust, N. E. Q. et al., 2009, AJ, 137, 246

\bibitem[\protect\citeauthoryear{Puls et al.}{2018}]{puls:18}
Puls, A. A., Alves-Brito, A., Campos, F., Dias, B. \& Barbuy, B., 2018, MNRAS, 476, 690

\end{thebibliography}

\end{document}
